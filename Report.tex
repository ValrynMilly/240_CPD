\documentclass[10pt,a4paper]{report}
\usepackage[latin2]{inputenc}
\usepackage{amsmath}
\usepackage{amsfonts}
\usepackage{amssymb}
\usepackage{makeidx}
\usepackage{graphicx}
\author{Emiljano Kurbiba}
\title{CPD - Year 2 Second Semester Report}
\maketitle
\begin{document}
	\section*{Introduction}
	In this report, I will be reflecting on the second semester of my second year at Falmouth university I will be discussing my many weaknesses and how to improve on them. This report is very personal, it has been the most reflective and honest report of the two years I have been studying at Falmouth university. 
	
	\section*{Second Semester}
	I have defiantly improved over second semester, I know last semester I mentioned in this report that I over scope on projects and almost came to fail the projects overall, this year however I have improved because I aimed low on the scope of the project and when I?ve completed that project early I improve on it with further development of what time I had left.\\\\
	My attendance has improved this semester but due to family reasons I had to go home and miss out on some lectures however I have kept up with lessons and work through learning space which was a huge help this semester.
	
	\section*{My Faults In Different Domains}
	\subsection*{Focus - Dispositional Domain}
	The weekly reports helped in this section because tracking through each week I started to realise that I focus too much on something, in particular Comp 240 collaborative game project. This was a huge problem for me, I need to schedule my focus more because this project took up most of my time and attention. This was most likely because it was the most challenging of the modules, not because it was difficult but because I knew I could always do better and kept going. 
	
	\subsection*{Source Control \& Project management}
	Although this semester my projects saw an improvement on this It was improved way late in development. I feel that I could have set up the git repository for 240 quicker and committed more other projects. Project management and Agile practices like Trello board and stand up meetings could be improved because it?s not at intermediate level of commitment. 
	As you can see I am lacking personal discipline on the dispositional domain, over the summer I will start a personal project and will regularly commit every day, a daily reminder will be set on my computer to commit on GitHub and update the projects Trello board.
	
	\subsection*{Enthusiasm - The affective domain}
	This one is much more personal, ever since I was very young I have never liked showing people the final finished product because I see that it can always be improved thus making me feel like it's never good enough (always liked showing people the best of me), the comp 240 project had demo's of which we had to show first years our games and frankly I was dreading it; I got so uncomfortable during the session that I started making jokes about how bad the game was and while people thought they were funny I was actually hating the project because it was never good enough for these people. Michael said considering it was 2 of us developing the project without BA students we did amazing but I didn't think so.
	\\\\
	I think this is one of the more serious faults of mine, I discussed this with some friends and they said that where I see faults in the project I should document \& evaluate them based on what I feel would make the game better, doing this over and over I would have a game that I would finally be proud of regardless of its success.
	
	\subsection*{The Thought Process of Blueprint Programming - Procedural Domain}
	While this semester has taught me a lot about blueprint programming however this is still much to learn, I mentioned in the weekly reports that some methods in my projects work but are incredibly impractical for the projects so over the summer I will work on a personal project of which I can learn and practice blueprint programming on an intermediate level. This will help massively since next year we?re in big teams working in unreal to create our final projects.
	
	\subsection*{Peer programming - Interpersonal Domain}
	This semester my partner and peers rarely peer programmed but when we did it we were really successful but rarely isn't good enough next year I'm going to push working in pair when we get stuck which is often rather than working on our own to overcome obstacles. This was not the only fault within the interpersonal domain, when we were presenting our collaborative projects to the first years I felt like I could have been way more professional when presenting rather than feeling like it?s not a big deal, this part of the domain is affected by the affective domain I must first overcome that fault in order to overcome this.
	
	\subsection*{The Status of Programming - Cognitive Domain}
	This year my programming has defiantly improved, the thought process and logical thinking behind programming is really fantastic I believe but my lack of syntax structure is lacking so I need to spend more and more time programming, currently I program 3 or 4 times a week. To improve this just takes practice so I'm going to try and push the weekly programming to everyday.
	
	\section*{Conclusion}
	This semester has been very personal, I think that because I failed with some modules in the past it's gotten me into a pattern of realising where I can improve. I think this way of thinking is only healthy since the path of progress starts with recognising where you are at fault. I really have enjoyed this semester; all modules have been incredibly fun. Working with my peers has been great and fully engaging, hopefully next year we will produce great games.
	
\end{document}